\documentclass[10pt,oneside]{book}
\makeatletter


%%%%%%%%%%%%%%%%%%%%%%%%%%%%%%%%%%%%%%%%%%%%%%%%%%%%%%%%%%%%%%%%%%%%%%%%%%%
% Lengths
%%%%%%%%%%%%%%%%%%%%%%%%%%%%%%%%%%%%%%%%%%%%%%%%%%%%%%%%%%%%%%%%%%%%%%%%%%%

\setlength{\parskip}{2pt plus 7pt minus 2pt}
%\setlength{\parskip}{3pt plus 6pt minus 3pt}
\newlength{\myskip}
\setlength{\myskip}{2.0ex plus 1.5ex minus 1.0ex}
%\setlength{\myskip}{3.0ex plus 1.5ex minus 1.5ex}
\newlength{\myexerciseskip}
\raggedbottom
%\addtolength{\topskip}{0pt plus 10pt} %http://www.tex.ac.uk/FAQ-nopagebrk.html


%%%%%%%%%%%%%%%%%%%%%%%%%%%%%%%%%%%%%%%%%%%%%%%%%%%%%%%%%%%%%%%%%%%%%%%%%%%
% Page Geometry
%%%%%%%%%%%%%%%%%%%%%%%%%%%%%%%%%%%%%%%%%%%%%%%%%%%%%%%%%%%%%%%%%%%%%%%%%%%

\usepackage[letterpaper, top=0.75in,
bottom=0.50in, outer=0.75in, textwidth=5.00in, marginparsep=0.25in,
marginparwidth=1.875in, reversemp]{geometry}


%%%%%%%%%%%%%%%%%%%%%%%%%%%%%%%%%%%%%%%%%%%%%%%%%%%%%%%%%%%%%%%%%%%%%%%%%%%
% Colors
%%%%%%%%%%%%%%%%%%%%%%%%%%%%%%%%%%%%%%%%%%%%%%%%%%%%%%%%%%%%%%%%%%%%%%%%%%%

\usepackage[usenames,dvipsnames,svgnames]{xcolor} % for colors
\definecolor{ocre}{RGB}{0,135,255} % main color
\definecolor{maincolor}{RGB}{0,145,255} % main color


%%%%%%%%%%%%%%%%%%%%%%%%%%%%%%%%%%%%%%%%%%%%%%%%%%%%%%%%%%%%%%%%%%%%%%%%%%%
% Fonts
%%%%%%%%%%%%%%%%%%%%%%%%%%%%%%%%%%%%%%%%%%%%%%%%%%%%%%%%%%%%%%%%%%%%%%%%%%%

\usepackage[utf8]{inputenc} % Required for including letters with accents
\usepackage[sc]{mathpazo} % Palatino
\usepackage[12pt]{moresize}
\usepackage{booktabs}
\usepackage[T1]{fontenc}
\usepackage[semibold]{raleway}
\usepackage[protrusion=true,expansion=true,kerning=true,spacing=true,tracking=true,final]{microtype}
\microtypecontext{spacing=nonfrench}



%%%%%%%%%%%%%%%%%%%%%%%%%%%%%%%%%%%%%%%%%%%%%%%%%%%%%%%%%%%%%%%%%%%%%%%%%%%
% INDEX
%%%%%%%%%%%%%%%%%%%%%%%%%%%%%%%%%%%%%%%%%%%%%%%%%%%%%%%%%%%%%%%%%%%%%%%%%%%


\usepackage{makeidx}
\usepackage[indentunit=1.00em, columns=3, font=small]{idxlayout}
\makeindex


%%%%%%%%%%%%%%%%%%%%%%%%%%%%%%%%%%%%%%%%%%%%%%%%%%%%%%%%%%%%%%%%%%%%%%%%%%%
% PAGE HEADERS
%%%%%%%%%%%%%%%%%%%%%%%%%%%%%%%%%%%%%%%%%%%%%%%%%%%%%%%%%%%%%%%%%%%%%%%%%%%


\usepackage{ifthen}
\usepackage{fancyhdr} % Required for header and footer configuration

\AtBeginDocument{%
    \fancyhfoffset[R]{0in}
    \fancyhfoffset[L]{2.00in} % fixes offset due to margin
}

\pagestyle{fancy}

% for vertical bars in headers
\newcommand{\vbarr}{\hspace{4.5pt}\rule[-0.5em]{0.5pt}{1.5em}\hspace{4.5pt}}

\renewcommand{\chaptermark}[1]{\markboth{\normalsize\bfseries\chaptername
\thechapter\vbarr#1}{}} % Chapter text font settings

\renewcommand{\sectionmark}[1]{\markright{\normalsize\textbf{\thesection}\vbarr#1}{}} % Section text font settings
\fancyhf{}


\fancyhead[R]{\ifthenelse{\isodd{\value{page}}}{\sffamily\rightmark\hspace{5ex}\large\thepage}{}}
\fancyhead[L]{\ifthenelse{\isodd{\value{page}}}{}{\hspace{-0.125in}\sffamily\large\thepage\hspace{5ex} \leftmark}}


\renewcommand{\headrulewidth}{0pt}
\renewcommand{\footrulewidth}{0pt} % Removes the rule in the footer

\addtolength{\headheight}{2.5pt} % Increase the spacing around the header slightly

\fancypagestyle{plain}{\fancyhead{}\renewcommand{\headrulewidth}{0pt}} % Style for when a plain pagestyle is specified

\fancypagestyle{fullwidth}{%
  \fancyhfoffset[L]{0in}
  \fancyhfoffset[R]{0in}
  \newgeometry{margin=0.75in}
}



% Removes the header from odd empty pages at the end of chapters
\renewcommand{\cleardoublepage}{%
\clearpage\ifodd\c@page\else
\hbox{}
\vspace*{\fill}
\thispagestyle{empty}
\newpage
\fi}


%%%%%%%%%%%%%%%%%%%%%%%%%%%%%%%%%%%%%%%%%%%%%%%%%%%%%%%%%%%%%%%%%%%%%%%%%%%
% Table of Contents Styling
%%%%%%%%%%%%%%%%%%%%%%%%%%%%%%%%%%%%%%%%%%%%%%%%%%%%%%%%%%%%%%%%%%%%%%%%%%%


\usepackage{titletoc} % Required for manipulating the table of contents
\contentsmargin{0cm} % Removes the default margin


% Part text styling
\titlecontents{part}[0cm]
{\addvspace{20pt}\centering\large\bfseries}
{}
{}
{}

% Chapter text styling
\titlecontents{chapter}[1.00cm] % Indentation
{\addvspace{5pt}\Large\sffamily\bfseries} % Spacing and font options for chapters
{\color{maincolor}\contentslabel[\Large\thecontentslabel]{1.00cm}\color{maincolor}} % Chapter number
{\color{maincolor}}
{\color{maincolor}\Large\;\titlerule*[.5pc]{.}\;\thecontentspage} % Page number

% Section text styling
\titlecontents{section}[1.00cm] % Indentation
{\addvspace{0pt}\sffamily\bfseries} % Spacing and font options for sections
{\contentslabel[\thecontentslabel]{1.00cm}} % Section number
{}
{\hfill\color{black}\thecontentspage} % Page number
[]

% Subsection text styling
\titlecontents{subsection}[1.00cm] % Indentation
{\addvspace{0pt}\sffamily\small} % Spacing and font options for subsections
{} % Subsection number
{}
{\ \titlerule*[.5pc]{.}\;\thecontentspage} % Page number
[]

%\titlecontents{chapter}[1.25cm]{\addvspace{1pc}\bfseries}{\contentslabel{5em}}{}
    %{\titlerule*[0.3pc]{.}\contentspage}

% makes partial table of contents
\newcommand{\chaptertoc}{%
\startcontents[chapters]
\vspace{3pt}
\printcontents[chapters]{}{1}{\setcounter{tocdepth}{2}}
\vspace{3pt}
}

% makes partial table of contents
\newcommand{\sectiontoc}{%
\startcontents[section]
\vspace{3pt}
\printcontents[section]{}{2}{}
\vspace{3pt}
}


%%%%%%%%%%%%%%%%%%%%%%%%%%%%%%%%%%%%%%%%%%%%%%%%%%%%%%%%%%%%%%%%%%%%%%%%%%%
% Headings
%%%%%%%%%%%%%%%%%%%%%%%%%%%%%%%%%%%%%%%%%%%%%%%%%%%%%%%%%%%%%%%%%%%%%%%%%%%


\usepackage{titlesec}

\newcommand\filltoend{\leavevmode{\unskip
  \leaders\hrule height.6ex depth\dimexpr-.5ex+0.4pt\hfill\hbox{}%
  \parfillskip=0pt\endgraf}}

%% \titleformat{<command>}[<shape>]{<format>}{<label>}{<sep>}{<before-code>}[<after-code>]

\titleformat{\chapter} %{<command>}[<shape>]
  {\sffamily\huge\bfseries} %{<format>}
  {\color{maincolor}{\chaptertitlename~\thechapter}} %{<label>}
  {8pt} %{<sep>}
  {} %{<before-code>}
  [] %{<after-code>}

\titleformat{\section} %{<command>}
  {\sffamily\LARGE\bfseries} %{<format>}
  {\color{maincolor}{\thesection}} %{<label>}
  {8pt} %{<sep>}
  {} %{<before-code>}
  [] %{<after-code>}

\titleformat{\subsection} %{<command>}
  {\sffamily\large\bfseries} %{<format>}
  {} %{<label>}
  {0ex} %{<sep>}
  {} %{<before-code>}

\titleformat{\subsubsection}
  {\sffamily\normalsize\bfseries}{\thesubsubsection}{1ex}{}
\titleformat{\paragraph}[runin]
  {\sffamily\large\bfseries\color{maincolor}}{\theparagraph}{1em}{}[\hspace*{0.5ex}\filltoend\vspace{-0.5em}]
\titleformat{\subparagraph}
  {\sffamily\LARGE\bfseries}
  {\color{maincolor}{\thesection}}{8pt}{}[]

\titlespacing*{\chapter} {-2.125in}{0pt}{9pt}
\titlespacing*{\section} {-2.125in}{3.5ex plus 1ex minus .2ex}{3.3ex plus .2ex}
\titlespacing*{\subsection} {0pt}{3.25ex plus 1ex minus .2ex}{1.5ex plus .2ex}
\titlespacing*{\subsubsection}{0pt}{3.25ex plus 1ex minus .2ex}{0.5ex plus .2ex}
\titlespacing*{\paragraph} {0pt}{3.25ex plus 1ex minus .2ex}{1em}
%\titlespacing*{\subparagraph} {\parindent}{3.25ex plus 1ex minus .2ex}{1em}
\titlespacing*{\subparagraph} {-0in}{3.5ex plus 1ex minus .2ex}{2.3ex plus .2ex}

\renewcommand{\bottomtitlespace}{2.5in}
%\newcommand{\sectionbreak}{\clearpage}
%\newcommand{\chapterbreak}{\cleardoublepage}


%%%%%%%%%%%%%%%%%%%%%%%%%%%%%%%%%%%%%%%%%%%%%%%%%%%%%%%%%%%%%%%%%%%%%%
% List Definitions and Styles
%%%%%%%%%%%%%%%%%%%%%%%%%%%%%%%%%%%%%%%%%%%%%%%%%%%%%%%%%%%%%%%%%%%%%%


\usepackage[inline]{enumitem} % customize lists

\setlist[enumerate,1]{
  label={\bfseries\arabic*.},
  ref={\bfseries\arabic*.},
  beginpenalty=2000, midpenalty=-1000, endpenalty=-1000,
  topsep={0pt plus 0pt minus 0pt},
  itemsep={2pt plus 1pt minus 2pt},
  wide, leftmargin=\parindent,
  labelwidth=!, labelindent=0pt
}


\newlist{benumerate}{enumerate}{1}
\setlist[benumerate]{label=\textbf{\alph*.},
  ref={\alph*},
  beginpenalty=2000, midpenalty=-1000, endpenalty=-1000,
  topsep={0pt plus 0pt minus 0pt},
  itemsep={2pt plus 1pt minus 2pt},
  wide, leftmargin=\parindent,
  labelwidth=!, labelindent=0pt
}


\newlist{eenumerate}{enumerate}{1}
\setlist[eenumerate]{label=\textbf{\arabic*.},
  ref={\arabic*},
  topsep=1pt, parsep=2pt,
  itemsep={2pt plus 4pt minus 2pt},
  wide, labelwidth=!, labelindent=0pt,
  leftmargin=18pt
}


\newlist{ienumerate}{enumerate*}{1}
\setlist[ienumerate,1]{%
  label=\textbf{\alph*.},
  topsep={2pt plus 1pt minus 1pt},
  afterlabel={~~~},
  itemjoin={\hfill}, after={\hfill\hfill},
  before={\\*}
}


%%%%%%%%%%%%%%%%%%%%%%%%%%%%%%%%%%%%%%%%%%%%%%%%%%%%%%%%%%%%%%%%%%%%%%
% THEOREM STYLES
%%%%%%%%%%%%%%%%%%%%%%%%%%%%%%%%%%%%%%%%%%%%%%%%%%%%%%%%%%%%%%%%%%%%%%


\usepackage{amsmath,amsthm} % For math equations
\usepackage{thmtools} % For math equations

\renewcommand{\qedsymbol}{\rotatebox[origin=c]{45}{\rule{4pt}{4pt}}}% Optional qed square
\declaretheoremstyle[
  spaceabove=\myskip,
  spacebelow=\myskip,
  headfont=\small\sffamily\bfseries,
  notefont=\small\sffamily\bfseries,
  notebraces={~---~}{},
  bodyfont=\normalfont,
  postheadspace=0.25em,
  postheadhook={~},
  qed=\qedsymbol,
]{solutionstyle}


\declaretheoremstyle[
spaceabove=0pt,
spacebelow=0pt,
  headfont=\small\sffamily\bfseries,
  notefont=\small\sffamily\bfseries,
  notebraces={---~}{},
  bodyfont=\normalfont,
  postheadspace=0.5em,
  ]{examplestyle}

\declaretheoremstyle[
spaceabove=0pt,
spacebelow=0pt,
  headfont=\small\sffamily\bfseries\color{ocre},
  notefont=\small\sffamily\bfseries,
  notebraces={---~}{},
  bodyfont=\normalfont,
  postheadspace=0.5em,
  ]{exercisestyle}

\newtheoremstyle{thmstyleblackwithspace}% ⟨name ⟩
{\myskip}% space above
{\myskip}% space below
{\normalfont}% body font
{0pt}% indent amount
{\small\sffamily\bfseries}% theorem head font
{}% Punctuation after theorem head
{.5em}% space after theorem head
{\thmname{#1}\thmnumber{~#2}\thmnote{~---~#3.}}% theorem head spec

\newtheoremstyle{thmstyleblack}% ⟨name ⟩
{0pt}% space above
{0pt}% space below
{\normalfont}% body font
{0pt}% indent amount
{\small\sffamily\bfseries}% theorem head font
{}% Punctuation after theorem head
{.5em}% space after theorem head
{\thmname{#1}\thmnumber{~#2}\thmnote{~---~#3.}}% theorem head spec

\newtheoremstyle{nonumberingwithqed}% ⟨name ⟩
{0pt}% space above
{0pt}% space below
{\normalfont}% body font
{0pt}% indent amount
{\small\sffamily\bfseries}% theorem head font
{}% Punctuation after theorem head
{.5em}% space after theorem head
{\thmname{#1.}\thmnumber{}\thmnote{}}% theorem head spec

% Defines the theorem text style for each type of theorems
\declaretheorem[name=Solution, style=solutionstyle, numbered=no]{solution}
\declaretheorem[name=Example, style=examplestyle, numbered=yes,
    parent=chapter]{example}
\declaretheorem[name=Exercise, style=exercisestyle, numbered=yes,
    parent=chapter]{exercise}
%\renewcommand*{\theexercise}{\arabic{exercise}} % change numbering style
\declaretheorem[style=solutionstyle, numbered=no, name=\textit{Proof}]{proofX}
\let\proof\proofX
%\declaretheorem[style=solutionstyle, numbered=no, name=Solution]{solutionX}
\newcounter{dummy}
\numberwithin{dummy}{chapter}
\theoremstyle{thmstyleblack}
\newtheorem{theorem}[dummy]{Theorem}
\newtheorem{definition}{Definition}[chapter]
\newtheorem{remark}{Remark}[chapter]
\theoremstyle{thmstyleblackwithspace}
%\newtheorem{example}{Example}[chapter]


%%%%%%%%%%%%%%%%%%%%%%%%%%%%%%%%%%%%%%%%%%%%%%%%%%%%%%%%%%%%%%%%%%%%%%
% DEFINITION OF COLORED BOXES
%%%%%%%%%%%%%%%%%%%%%%%%%%%%%%%%%%%%%%%%%%%%%%%%%%%%%%%%%%%%%%%%%%%%%%


\RequirePackage[framemethod=default]{mdframed} % required for colored boxes

%% Theorem box
\mdfdefinestyle{tBox}{skipabove=\myskip,
skipbelow=\myskip,
nobreak=true,
backgroundcolor=black!5,
linecolor=maincolor,
linewidth=1.50pt,
innerleftmargin=5pt,
innerrightmargin=5pt,
innertopmargin=5pt,
leftmargin=0cm,
rightmargin=0cm,
innerbottommargin=5pt}


%% Definition box
\mdfdefinestyle{dBox}{skipabove=\myskip,
skipbelow=\myskip,
nobreak=true,
rightline=true,
leftline=true,
topline=true,
bottomline=true,
backgroundcolor=maincolor!10,
linecolor=maincolor,
innerleftmargin=5pt,
innerrightmargin=5pt,
innertopmargin=5pt,
innerbottommargin=5pt,
leftmargin=0cm,
rightmargin=0cm,
linewidth=1.5pt}

%% Exercise box
\mdfdefinestyle{eBox}{skipabove=\myskip,
skipbelow=\myskip,
rightline=false,
leftline=true,
topline=false,
bottomline=false,
backgroundcolor=gray!10,
linecolor=maincolor,
innerleftmargin=5pt,
innerrightmargin=5pt,
innertopmargin=5pt,
innerbottommargin=5pt,
leftmargin=0cm,
rightmargin=0cm,
linewidth=4pt,
innerbottommargin=0pt}

%% Corollary box
\mdfdefinestyle{cBox}{skipabove=\myskip,
skipbelow=\myskip,
nobreak=true,
rightline=false,
leftline=true,
topline=false,
bottomline=false,
linecolor=gray,
backgroundcolor=black!5,
innerleftmargin=5pt,
innerrightmargin=5pt,
innertopmargin=5pt,
leftmargin=0cm,
rightmargin=0cm,
linewidth=4pt,
innerbottommargin=5pt}

\mdfdefinestyle{rBox}{skipabove=\myskip,
skipbelow=\myskip,
roundcorner=5pt,
outerlinewidth=0,
nobreak=true,
backgroundcolor=maincolor!10,
linecolor=maincolor,
innerleftmargin=7pt,
innerrightmargin=7pt,
innertopmargin=7pt,
innerbottommargin=7pt,
leftmargin=0pt,
rightmargin=0pt,
linewidth=1pt
}

\mdfdefinestyle{sBox}{skipabove=\myskip,
skipbelow=\myskip,
nobreak=true,
hidealllines,
innerleftmargin=0pt,
innerrightmargin=0pt,
innertopmargin=0pt,
innerbottommargin=0pt,
leftmargin=0pt,
rightmargin=0pt,
linewidth=0pt
}


% surround theorems with boxes
\surroundwithmdframed[style=tBox]{theorem}
\surroundwithmdframed[style=dBox]{definition}
\surroundwithmdframed[style=eBox]{exercise}
\surroundwithmdframed[style=sBox]{example}

% use this to surround solutions with a samepage environment
%\newenvironment{solution}{\begin{samepage}\begin{solutionX}}{\end{solutionX}\end{samepage}}


%%%%%%%%%%%%%%%%%%%%%%%%%%%%%%%%%%%%%%%%%%%%%%%%%%%%%%%%%%%%%%%%%%%%%%%%%%%
% Widefigure and Objectives Environments
%%%%%%%%%%%%%%%%%%%%%%%%%%%%%%%%%%%%%%%%%%%%%%%%%%%%%%%%%%%%%%%%%%%%%%%%%%%
\usepackage{environ}


\newenvironment{widefigure}[1][htbp]{
\begin{figure}[#1]
  \begin{adjustwidth}{-2.00in}{-0in}
}
{
\end{adjustwidth}
\end{figure}
}


% Use this environment if you don't want objectives in the margins
% \NewEnviron{objectives}[1]{%
% \begin{adjustwidth}{-2.125in}{0in}

% \noindent{\sffamily\textbf{\Large Objectives}}

% \noindent%
% #1
% \vspace{-3mm}
% \begin{multicols}{2}
% \begin{itemize}[nosep,leftmargin=10.0pt]
%     \BODY
% \end{itemize}
% \end{multicols}
% \end{adjustwidth}

% \marginpar{\rule{1cm}{0in}}
% }


% Use this environment if you want objectives in the margins
\NewEnviron{objectives}[1]{%
  \marginpar{%
    \vspace{-0.1em}
  \noindent\textbf{\sffamily\Large Objectives}\\[2pt]
    #1
\begin{itemize}[nosep,leftmargin=10.0pt]
    \BODY
\end{itemize}
  \vspace{1em minus 1em}
  }
}



\let\oldsection\section
\renewcommand{\section}[1]{\oldsection{#1}\marginpar{\rule{1cm}{0in}}}


%%%%%%%%%%%%%%%%%%%%%%%%%%%%%%%%%%%%%%%%%%%%%%%%%%%%%%%%%%%%%%%%%%%%%%%%%%%
% Miscellaneous Packages
%%%%%%%%%%%%%%%%%%%%%%%%%%%%%%%%%%%%%%%%%%%%%%%%%%%%%%%%%%%%%%%%%%%%%%%%%%%


\usepackage{docmute} % for muting preamble of input files
% https://en.wikibooks.org/wiki/LaTeX/Page_Layout
% \usepackage[defaultlines=3,all]{nowidow}
\usepackage{graphicx} % Required for including pictures
\graphicspath{{pictures/}} % Specifies where pictures are stored
\usepackage{multicol}
\usepackage{float} % for H option in float placement
\usepackage{sidenotes}  % for marginfigure/margintable environment
\usepackage[font=small,labelfont={sf}]{caption}
\usepackage{subcaption}
\DeclareCaptionStyle{marginfigure}{font=footnotesize,labelfont={sf}}
\DeclareCaptionStyle{margintable}{font=footnotesize,labelfont={sf}}
\DeclareCaptionStyle{sidecaption}{labelfont={sf}}
\let\footnote\marginnote  %% marginpars in environments like align don't work
\usepackage{colortbl}
\usepackage{cancel}
\usepackage[bookmarksnumbered,linkbordercolor={1 1 1}]{hyperref} % for hyperlinks within document




%\input{glyphtounicode}
%\pdfgentounicode=1

\makeatother





\geometry{showframe}
\overfullrule=5pt


\pagestyle{fancy}


\begin{document}

\chapter{Examples of Usage}


\section{Functions and Environments}


\begin{objectives}{In this section, we learn how to use basic environments. We will}
  \item
    learn how to use the benumerate environment for lists.
  \item
    learn how to use the ienumerate environment for lists.
  \item
    learn how to use the theorem, definition, example, solution, exercise, and remark environments
\end{objectives}


\noindent
Some text\ldots


This is how you refer to Table~\ref{tab:tableexample}, which is defined below in
the ``.tex'' file.


\begin{marginfigure}
    \includegraphics[width=\linewidth]{placeholder.jpg}
    \caption{This is a how you put a figure in the margin.}
    \label{fig:marginfig2}
\end{marginfigure}


This is how you make a list to denote an example with parts. This paragraph will
be indented.  It uses the environment \texttt{benumerate} to start the list
environment on the second level and with bold styling. \index{benumerate}
\begin{benumerate}
  \item first
  \item second
  \item third
\end{benumerate}


If your list is small enough to fit one line, use the \texttt{ienumerate}
environment.\index{ienumerate}
\begin{ienumerate}
  \item first
  \item second
  \item third
\end{ienumerate}


\begin{theorem}[Pythagorean Theorem]
  \index{Pythagorean Theorem}
  For any right triangle with legs $a$, $b$  and hypotenuse $c$, the following
  is always true:
  \begin{align*}
    c^2 = a^2 + b^2
  \end{align*}
\end{theorem}


\begin{definition}[Triangle]
  A \textbf{triangle} is a polygon with three sides.
  \index{Triangle}
\end{definition}


\begin{example}[Optional Name of Example]
  Text goes here.  Solve the equation $x +1 = 4$.
\end{example}


\begin{solution}
  To solve we subtract $1$.
  \begin{align*}
    x + 1 &= 3     \\
        x &= 3 - 1 \\
        x &= 2
  \end{align*}
  Do not leave blank lines at the end of the solution environment.
  Otherwise, the black square will be misplaced.
\end{solution}


\begin{example}
  Find the derivatives of the following functions.
  \begin{ienumerate}
    \item
      $f(x) = 3x^2$
    \item
      $g(x) = e^{3x}$
    \item
      $h(x) = \cos(5x)$
  \end{ienumerate}
\end{example}


\begin{solution}
The derivatives are as follows.
  \begin{ienumerate}
    \item
      $f'(x) = 6x$
    \item
      $g'(x) = 3e^{3x}$
    \item
      $h'(x) = -5\sin(5x)$
  \end{ienumerate}
\end{solution}


\begin{exercise}[Optional Name]
  I would imagine this environment will be rarely used.
\end{exercise}


\begin{remark}
  This is an example of the remark environment.
  I would imagine this environment will be rarely used.
\end{remark}



%\begin{corollary}[Pythagorean Theorem]
  %For any right triangle with legs $a$, $b$  and hypotenuse $c$, the following
  %is always true:
  %\begin{align*}
    %c^2 = a^2 + b^2
  %\end{align*}
%\end{corollary}





\newpage


\section{Positioning Figures and Tables}


\begin{objectives}{In this section, we learn how to create useful layouts. We will}
    \item
      learn to place figures within the main column of text.
    \item
      learn to place figures that go into the margin.
    \item
      learn to place figures within the margin.
    \item
      learn to place figures side-by-side in different ways.
  \end{objectives}


Dealing with figures in \LaTeX is not easy.  This is a collection of different 
ways to position figures.  Figures are usually intended to be ``floating,'' which
means that \LaTeX\ has discretion on where to place them within the document.


This is how you include an image, that is not intended to be a figure.

\includegraphics[width=0.9\linewidth]{placeholder.jpg}

The following will have a non-floating figure. The figure will not move from the position it's placed

\noindent
\begin{minipage}{0.9\linewidth}
  \centering
  \includegraphics[width=0.9\linewidth]{placeholder.jpg}
  \captionof{figure}{A non-floating figure. The figure will not move from the position it's placed}
  \label{fig:notfloatfig}
\end{minipage}



\begin{margintable}
  \centering
  \caption{This is a caption for a table that will be placed in the margin.}
  \label{tab:margintab}
  \begin{tabular}{rrr}
    \toprule
    x & y & z \\
    \midrule
    1 & 2 & 3 \\
    \bottomrule
  \end{tabular}
\end{margintable}


\begin{table}[bhpt]%optional location preferences: bottom, here, own page, top
  \centering
  \caption{This is a caption for a table that will be placed in the main text.}
  \label{tab:tableexample}
  \begin{tabular}{rrr}
    \toprule
    x & y & z \\
    \midrule
    1 & 2 & 3 \\
    \bottomrule
  \end{tabular}
\end{table}


\begin{marginfigure}
  \includegraphics[width=\linewidth]{placeholder.jpg}
  \caption{This is a how you put a figure in the margin.}
  \label{fig:marginfig}
\end{marginfigure}

\begin{figure}
  \centering
  \includegraphics[width=\linewidth]{placeholder.jpg}
  \caption{This is a how you put a floating figure in the main text.}
  \label{fig:floatingfig}
\end{figure}

\begin{widefigure}
  \centering
  \includegraphics[width=\linewidth]{placeholder.jpg}
  \caption{This is a how you put a \textbf{wide} floating figure in the main text.}
\end{widefigure}


\begin{figure}
    \centering
    \begin{subfigure}[b]{0.47\linewidth}
      \includegraphics[width=\linewidth]{placeholder.jpg}
      \subcaption{This is a subcaption.}
    \end{subfigure}
    \quad
    \begin{subfigure}[b]{0.47\linewidth}
      \includegraphics[width=\linewidth]{placeholder.jpg}
      \subcaption{This is another subcaption.}
    \end{subfigure}
  \caption{This is a how you put subfigures in a figure.}
\end{figure}


\begin{widefigure}
    \centering
    \begin{subfigure}[b]{0.45\linewidth}
      \includegraphics[width=\linewidth]{placeholder.jpg}
      \subcaption{This is a subcaption.}
    \end{subfigure}
    \qquad
    \begin{subfigure}[b]{0.45\linewidth}
      \includegraphics[width=\linewidth]{placeholder.jpg}
      \subcaption{This is another subcaption.}
    \end{subfigure}
  \caption{This is a how you put subfigures in a wide figure.}
\end{widefigure}



\begin{figure}
    \centering
    \begin{subfigure}[b]{0.45\linewidth}
      \includegraphics[width=\linewidth]{placeholder.jpg}
      \subcaption*{There is no subcaption heading here.}
    \end{subfigure}
    \qquad
    \begin{subfigure}[b]{0.45\linewidth}
      \includegraphics[width=\linewidth]{placeholder.jpg}
      \subcaption*{There is no subcaption heading here, either.}
    \end{subfigure}
  \caption{This is a how you put subfigures without a subcaption heading.}
\end{figure}



\begin{widefigure}
\centering
\begin{minipage}{.48\linewidth}
  \centering
  \includegraphics[width=0.9\linewidth]{placeholder.jpg}
  \captionof{figure}{This is how you put side-by-side figures.}
  \label{fig:graph1}
\end{minipage}%
\qquad
\begin{minipage}{.48\linewidth}
  \centering
  \includegraphics[width=0.9\linewidth]{placeholder.jpg}
  \captionof{figure}{This is how you put side-by-side figures.}
  \label{fig:graph2}
\end{minipage}
\end{widefigure}



\noindent
\begin{minipage}{\textwidth}
  This is a non-floating side-by-side figure.

\flushleft
\begin{minipage}{.48\linewidth}
  \centering
  \includegraphics[width=1.0\linewidth]{placeholder.jpg}
  \captionof{figure}{Side-by-side figures.}
  \label{fig:graph1}
\end{minipage}%
\quad
\begin{minipage}{.48\linewidth}
  \centering
  \includegraphics[width=1.0\linewidth]{placeholder.jpg}
  \captionof{figure}{Side-by-side figures.}
  \label{fig:graph2}
\end{minipage}
\end{minipage}


\noindent
\begin{minipage}{\textwidth}
  This is a non-floating table next to a figure.

\flushleft
\begin{minipage}{.48\linewidth}
  \centering
  \captionof{table}{This is a caption for a table.}
  \label{tab:tableexample3}
  \begin{tabular}{rrr}
    \toprule
    x & y & z \\
    \midrule
    1 & 2 & 3 \\
    \bottomrule
  \end{tabular}
\end{minipage}
\quad
\begin{minipage}{.48\linewidth}
  \centering
  \includegraphics[width=1.0\linewidth]{placeholder.jpg}
  \captionof{figure}{Side-by-side figures.}
  \label{fig:graph1}
\end{minipage}%
\end{minipage}



\begin{figure}
  This is a floating table next to a figure.

\flushleft
\begin{minipage}{.48\linewidth}
  \centering
  \captionof{table}{This is a caption for a table.}
  \label{tab:tableexample4}
  \begin{tabular}{rrr}
    \toprule
    x & y & z \\
    \midrule
    1 & 2 & 3 \\
    \bottomrule
  \end{tabular}
\end{minipage}
\quad
\begin{minipage}{.48\linewidth}
  \centering
  \includegraphics[width=1.0\linewidth]{placeholder.jpg}
  \captionof{figure}{Side-by-side figures.}
  \label{fig:graph1}
\end{minipage}%
\end{figure}



\begin{widefigure}
  \begin{minipage}[t]{0.25\linewidth}
    \mbox{}\\
    \captionof{figure}{%
      This is a figure with a caption in the margin.
    }
    \label{fig:blah}
  \end{minipage}
  \hfill
  \begin{minipage}[t]{0.715\linewidth}
    \mbox{}\\
    \includegraphics[width=\linewidth]{placeholder.jpg}
  \end{minipage}
\end{widefigure}



\begin{figure}
  \begin{minipage}[t]{0.25\linewidth}
    \mbox{}\\
    \captionof{figure}{%
      This is a figure with a caption in the margin.
    }
    \label{fig:blah}
  \end{minipage}
  \hfill
  \begin{minipage}[t]{0.715\linewidth}
    \mbox{}\\
    \includegraphics[width=\linewidth]{placeholder.jpg}
  \end{minipage}
\end{figure}



\end{document}
