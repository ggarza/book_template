\makeatletter


%%%%%%%%%%%%%%%%%%%%%%%%%%%%%%%%%%%%%%%%%%%%%%%%%%%%%%%%%%%%%%%%%%%%%%%%%%%
% Lengths
%%%%%%%%%%%%%%%%%%%%%%%%%%%%%%%%%%%%%%%%%%%%%%%%%%%%%%%%%%%%%%%%%%%%%%%%%%%

\setlength{\parskip}{2pt plus 7pt minus 2pt}
%\setlength{\parskip}{3pt plus 6pt minus 3pt}
\newlength{\myskip}
\setlength{\myskip}{2.0ex plus 1.5ex minus 1.0ex}
%\setlength{\myskip}{3.0ex plus 1.5ex minus 1.5ex}
\newlength{\myexerciseskip}
\raggedbottom
%\addtolength{\topskip}{0pt plus 10pt} %http://www.tex.ac.uk/FAQ-nopagebrk.html


%%%%%%%%%%%%%%%%%%%%%%%%%%%%%%%%%%%%%%%%%%%%%%%%%%%%%%%%%%%%%%%%%%%%%%%%%%%
% Page Geometry
%%%%%%%%%%%%%%%%%%%%%%%%%%%%%%%%%%%%%%%%%%%%%%%%%%%%%%%%%%%%%%%%%%%%%%%%%%%

\usepackage[letterpaper, top=0.75in,
bottom=0.50in, outer=0.75in, textwidth=5.00in, marginparsep=0.25in,
marginparwidth=1.875in, reversemp]{geometry}


%%%%%%%%%%%%%%%%%%%%%%%%%%%%%%%%%%%%%%%%%%%%%%%%%%%%%%%%%%%%%%%%%%%%%%%%%%%
% Colors
%%%%%%%%%%%%%%%%%%%%%%%%%%%%%%%%%%%%%%%%%%%%%%%%%%%%%%%%%%%%%%%%%%%%%%%%%%%

\usepackage[usenames,dvipsnames,svgnames]{xcolor} % for colors
\definecolor{ocre}{RGB}{0,135,255} % main color
\definecolor{maincolor}{RGB}{0,145,255} % main color


%%%%%%%%%%%%%%%%%%%%%%%%%%%%%%%%%%%%%%%%%%%%%%%%%%%%%%%%%%%%%%%%%%%%%%%%%%%
% Fonts
%%%%%%%%%%%%%%%%%%%%%%%%%%%%%%%%%%%%%%%%%%%%%%%%%%%%%%%%%%%%%%%%%%%%%%%%%%%

\usepackage[utf8]{inputenc} % Required for including letters with accents
\usepackage[sc]{mathpazo} % Palatino
\usepackage[12pt]{moresize}
\usepackage{booktabs}
\usepackage[T1]{fontenc}
\usepackage[semibold]{raleway}
\usepackage[protrusion=true,expansion=true,kerning=true,spacing=true,tracking=true,final]{microtype}
\microtypecontext{spacing=nonfrench}



%%%%%%%%%%%%%%%%%%%%%%%%%%%%%%%%%%%%%%%%%%%%%%%%%%%%%%%%%%%%%%%%%%%%%%%%%%%
% INDEX
%%%%%%%%%%%%%%%%%%%%%%%%%%%%%%%%%%%%%%%%%%%%%%%%%%%%%%%%%%%%%%%%%%%%%%%%%%%


\usepackage{makeidx}
\usepackage[indentunit=1.00em, columns=3, font=small]{idxlayout}
\makeindex


%%%%%%%%%%%%%%%%%%%%%%%%%%%%%%%%%%%%%%%%%%%%%%%%%%%%%%%%%%%%%%%%%%%%%%%%%%%
% PAGE HEADERS
%%%%%%%%%%%%%%%%%%%%%%%%%%%%%%%%%%%%%%%%%%%%%%%%%%%%%%%%%%%%%%%%%%%%%%%%%%%


\usepackage{ifthen}
\usepackage{fancyhdr} % Required for header and footer configuration

\AtBeginDocument{%
    \fancyhfoffset[R]{0in}
    \fancyhfoffset[L]{2.00in} % fixes offset due to margin
}

\pagestyle{fancy}

\renewcommand{\chaptermark}[1]{\markboth{\normalsize\bfseries\chaptername\ \thechapter\ \ #1}{}} % Chapter text font settings

\renewcommand{\sectionmark}[1]{\markright{\normalsize\thesection\hspace{5pt}#1}{}} % Section text font settings
\fancyhf{}


\fancyhead[R]{\ifthenelse{\isodd{\value{page}}}{\sffamily\rightmark\hspace{5ex}\thepage}{}}
\fancyhead[L]{\ifthenelse{\isodd{\value{page}}}{}{\hspace{-0.125in}\sffamily\thepage\hspace{5ex} \leftmark}}


\renewcommand{\headrulewidth}{0pt}
\renewcommand{\footrulewidth}{0pt} % Removes the rule in the footer

\addtolength{\headheight}{2.5pt} % Increase the spacing around the header slightly

\fancypagestyle{plain}{\fancyhead{}\renewcommand{\headrulewidth}{0pt}} % Style for when a plain pagestyle is specified

\fancypagestyle{fullwidth}{%
  \fancyhfoffset[L]{0in}
  \fancyhfoffset[R]{0in}
  \newgeometry{margin=0.75in}
}



% Removes the header from odd empty pages at the end of chapters
\renewcommand{\cleardoublepage}{%
\clearpage\ifodd\c@page\else
\hbox{}
\vspace*{\fill}
\thispagestyle{empty}
\newpage
\fi}


%%%%%%%%%%%%%%%%%%%%%%%%%%%%%%%%%%%%%%%%%%%%%%%%%%%%%%%%%%%%%%%%%%%%%%%%%%%
% Table of Contents Styling
%%%%%%%%%%%%%%%%%%%%%%%%%%%%%%%%%%%%%%%%%%%%%%%%%%%%%%%%%%%%%%%%%%%%%%%%%%%


\usepackage{titletoc} % Required for manipulating the table of contents
\contentsmargin{0cm} % Removes the default margin


% Part text styling
\titlecontents{part}[0cm]
{\addvspace{20pt}\centering\large\bfseries}
{}
{}
{}

% Chapter text styling
\titlecontents{chapter}[1.00cm] % Indentation
{\addvspace{5pt}\Large\sffamily\bfseries} % Spacing and font options for chapters
{\color{maincolor}\contentslabel[\Large\thecontentslabel]{1.00cm}\color{maincolor}} % Chapter number
{\color{maincolor}}
{\color{maincolor}\Large\;\titlerule*[.5pc]{.}\;\thecontentspage} % Page number

% Section text styling
\titlecontents{section}[1.00cm] % Indentation
{\addvspace{0pt}\sffamily\bfseries} % Spacing and font options for sections
{\contentslabel[\thecontentslabel]{1.00cm}} % Section number
{}
{\hfill\color{black}\thecontentspage} % Page number
[]

% Subsection text styling
\titlecontents{subsection}[1.00cm] % Indentation
{\addvspace{0pt}\sffamily\small} % Spacing and font options for subsections
{} % Subsection number
{}
{\ \titlerule*[.5pc]{.}\;\thecontentspage} % Page number
[]

%\titlecontents{chapter}[1.25cm]{\addvspace{1pc}\bfseries}{\contentslabel{5em}}{}
    %{\titlerule*[0.3pc]{.}\contentspage}

% makes partial table of contents
\newcommand{\chaptertoc}{%
\startcontents[chapters]
\vspace{3pt}
\printcontents[chapters]{}{1}{\setcounter{tocdepth}{2}}
\vspace{3pt}
}

% makes partial table of contents
\newcommand{\sectiontoc}{%
\startcontents[section]
\vspace{3pt}
\printcontents[section]{}{2}{}
\vspace{3pt}
}


%%%%%%%%%%%%%%%%%%%%%%%%%%%%%%%%%%%%%%%%%%%%%%%%%%%%%%%%%%%%%%%%%%%%%%%%%%%
% Headings
%%%%%%%%%%%%%%%%%%%%%%%%%%%%%%%%%%%%%%%%%%%%%%%%%%%%%%%%%%%%%%%%%%%%%%%%%%%


\usepackage{titlesec}

%% \titleformat{<command>}[<shape>]{<format>}{<label>}{<sep>}{<before-code>}[<after-code>]

\titleformat{\chapter} %{<command>}[<shape>]
  {\sffamily\huge\bfseries} %{<format>}
  {\color{maincolor}{\chaptertitlename~\thechapter}} %{<label>}
  {8pt} %{<sep>}
  {} %{<before-code>}
  [] %{<after-code>}

\titleformat{\section} %{<command>}
  {\sffamily\LARGE\bfseries} %{<format>}
  {\color{maincolor}{\thesection}} %{<label>}
  {8pt} %{<sep>}
  {} %{<before-code>}
  [] %{<after-code>}

\titleformat{\subsection} %{<command>}
  {\sffamily\large\bfseries} %{<format>}
  {} %{<label>}
  {0ex} %{<sep>}
  {} %{<before-code>}

\titleformat{\subsubsection}
  {\sffamily\normalsize\bfseries}{\thesubsubsection}{1ex}{}
\titleformat{\paragraph}[runin]
  {\sffamily\normalsize\bfseries}{\theparagraph}{1em}{}
\titleformat{\subparagraph}
  {\sffamily\LARGE\bfseries}
  {\color{maincolor}{\thesection}}{8pt}{}[]

\titlespacing*{\chapter} {-2.125in}{0pt}{9pt}
\titlespacing*{\section} {-2.125in}{3.5ex plus 1ex minus .2ex}{3.3ex plus .2ex}
\titlespacing*{\subsection} {0pt}{3.25ex plus 1ex minus .2ex}{1.5ex plus .2ex}
\titlespacing*{\subsubsection}{0pt}{3.25ex plus 1ex minus .2ex}{0.5ex plus .2ex}
\titlespacing*{\paragraph} {0pt}{3.25ex plus 1ex minus .2ex}{1em}
%\titlespacing*{\subparagraph} {\parindent}{3.25ex plus 1ex minus .2ex}{1em}
\titlespacing*{\subparagraph} {-0in}{3.5ex plus 1ex minus .2ex}{2.3ex plus .2ex}

\renewcommand{\bottomtitlespace}{2.5in}
%\newcommand{\sectionbreak}{\clearpage}
%\newcommand{\chapterbreak}{\cleardoublepage}


%%%%%%%%%%%%%%%%%%%%%%%%%%%%%%%%%%%%%%%%%%%%%%%%%%%%%%%%%%%%%%%%%%%%%%
% List Definitions and Styles
%%%%%%%%%%%%%%%%%%%%%%%%%%%%%%%%%%%%%%%%%%%%%%%%%%%%%%%%%%%%%%%%%%%%%%


\usepackage[inline]{enumitem} % customize lists

\setlist[enumerate,1]{
  label={\bfseries\arabic*.},
  ref={\bfseries\arabic*.},
  beginpenalty=2000, midpenalty=-1000, endpenalty=-1000,
  topsep={0pt plus 0pt minus 0pt},
  itemsep={2pt plus 1pt minus 2pt},
  wide, leftmargin=\parindent,
  labelwidth=!, labelindent=0pt
}


\newlist{benumerate}{enumerate}{1}
\setlist[benumerate]{label=\textbf{\alph*.},
  ref={\alph*},
  beginpenalty=2000, midpenalty=-1000, endpenalty=-1000,
  topsep={0pt plus 0pt minus 0pt},
  itemsep={2pt plus 1pt minus 2pt},
  wide, leftmargin=\parindent,
  labelwidth=!, labelindent=0pt
}


\newlist{eenumerate}{enumerate}{1}
\setlist[eenumerate]{label=\textbf{\arabic*.},
  ref={\arabic*},
  topsep=1pt, parsep=2pt,
  itemsep={2pt plus 4pt minus 2pt},
  wide, labelwidth=!, labelindent=0pt,
  leftmargin=18pt
}


\newlist{ienumerate}{enumerate*}{1}
\setlist[ienumerate,1]{%
  label=\textbf{\alph*.},
  topsep={2pt plus 1pt minus 1pt},
  afterlabel={~~~},
  itemjoin={\hfill}, after={\hfill\hfill},
  before={\\*}
}


%%%%%%%%%%%%%%%%%%%%%%%%%%%%%%%%%%%%%%%%%%%%%%%%%%%%%%%%%%%%%%%%%%%%%%
% THEOREM STYLES
%%%%%%%%%%%%%%%%%%%%%%%%%%%%%%%%%%%%%%%%%%%%%%%%%%%%%%%%%%%%%%%%%%%%%%


\usepackage{amsmath,amsthm} % For math equations
\usepackage{thmtools} % For math equations

\renewcommand{\qedsymbol}{\rule{4pt}{4pt}}% Optional qed square
\declaretheoremstyle[
  spaceabove=\myskip,
  spacebelow=\myskip,
  headfont=\small\sffamily\bfseries,
  notefont=\small\sffamily\bfseries,
  notebraces={~---~}{},
  bodyfont=\normalfont,
  postheadspace=0.25em,
  postheadhook={~},
  qed=\qedsymbol,
]{solutionstyle}


\declaretheoremstyle[
spaceabove=0pt,
spacebelow=0pt,
  headfont=\small\sffamily\bfseries,
  notefont=\small\sffamily\bfseries,
  notebraces={---~}{},
  bodyfont=\normalfont,
  postheadspace=0.5em,
  ]{examplestyle}

\declaretheoremstyle[
spaceabove=0pt,
spacebelow=0pt,
  headfont=\small\sffamily\bfseries\color{ocre},
  notefont=\small\sffamily\bfseries,
  notebraces={---~}{},
  bodyfont=\normalfont,
  postheadspace=0.5em,
  ]{exercisestyle}

\newtheoremstyle{thmstyleblackwithspace}% ⟨name ⟩
{\myskip}% space above
{\myskip}% space below
{\normalfont}% body font
{0pt}% indent amount
{\small\sffamily\bfseries}% theorem head font
{}% Punctuation after theorem head
{.5em}% space after theorem head
{\thmname{#1}\thmnumber{~#2}\thmnote{~---~#3.}}% theorem head spec

\newtheoremstyle{thmstyleblack}% ⟨name ⟩
{0pt}% space above
{0pt}% space below
{\normalfont}% body font
{0pt}% indent amount
{\small\sffamily\bfseries}% theorem head font
{}% Punctuation after theorem head
{.5em}% space after theorem head
{\thmname{#1}\thmnumber{~#2}\thmnote{~---~#3.}}% theorem head spec

\newtheoremstyle{nonumberingwithqed}% ⟨name ⟩
{0pt}% space above
{0pt}% space below
{\normalfont}% body font
{0pt}% indent amount
{\small\sffamily\bfseries}% theorem head font
{}% Punctuation after theorem head
{.5em}% space after theorem head
{\thmname{#1.}\thmnumber{}\thmnote{}}% theorem head spec

% Defines the theorem text style for each type of theorems
\declaretheorem[name=Solution, style=solutionstyle, numbered=no]{solution}
\declaretheorem[name=Example, style=examplestyle, numbered=yes,
    parent=chapter]{example}
\declaretheorem[name=Exercise, style=exercisestyle, numbered=yes,
    parent=chapter]{exercise}
%\renewcommand*{\theexercise}{\arabic{exercise}} % change numbering style
\declaretheorem[style=solutionstyle, numbered=no, name=\textit{Proof}]{proofX}
\let\proof\proofX
%\declaretheorem[style=solutionstyle, numbered=no, name=Solution]{solutionX}
\newcounter{dummy}
\numberwithin{dummy}{chapter}
\theoremstyle{thmstyleblack}
\newtheorem{theorem}[dummy]{Theorem}
\newtheorem{definition}{Definition}[chapter]
\newtheorem{remark}{Remark}[chapter]
\theoremstyle{thmstyleblackwithspace}
%\newtheorem{example}{Example}[chapter]


%%%%%%%%%%%%%%%%%%%%%%%%%%%%%%%%%%%%%%%%%%%%%%%%%%%%%%%%%%%%%%%%%%%%%%
% DEFINITION OF COLORED BOXES
%%%%%%%%%%%%%%%%%%%%%%%%%%%%%%%%%%%%%%%%%%%%%%%%%%%%%%%%%%%%%%%%%%%%%%


\RequirePackage[framemethod=default]{mdframed} % required for colored boxes

%% Theorem box
\mdfdefinestyle{tBox}{skipabove=\myskip,
skipbelow=\myskip,
nobreak=true,
backgroundcolor=black!5,
linecolor=maincolor,
linewidth=1.50pt,
innerleftmargin=5pt,
innerrightmargin=5pt,
innertopmargin=5pt,
leftmargin=0cm,
rightmargin=0cm,
innerbottommargin=5pt}


%% Definition box
\mdfdefinestyle{dBox}{skipabove=\myskip,
skipbelow=\myskip,
nobreak=true,
rightline=true,
leftline=true,
topline=true,
bottomline=true,
backgroundcolor=maincolor!10,
linecolor=maincolor,
innerleftmargin=5pt,
innerrightmargin=5pt,
innertopmargin=5pt,
innerbottommargin=5pt,
leftmargin=0cm,
rightmargin=0cm,
linewidth=1.5pt}

%% Exercise box
\mdfdefinestyle{eBox}{skipabove=\myskip,
skipbelow=\myskip,
rightline=false,
leftline=true,
topline=false,
bottomline=false,
backgroundcolor=gray!10,
linecolor=maincolor,
innerleftmargin=5pt,
innerrightmargin=5pt,
innertopmargin=5pt,
innerbottommargin=5pt,
leftmargin=0cm,
rightmargin=0cm,
linewidth=4pt,
innerbottommargin=0pt}

%% Corollary box
\mdfdefinestyle{cBox}{skipabove=\myskip,
skipbelow=\myskip,
nobreak=true,
rightline=false,
leftline=true,
topline=false,
bottomline=false,
linecolor=gray,
backgroundcolor=black!5,
innerleftmargin=5pt,
innerrightmargin=5pt,
innertopmargin=5pt,
leftmargin=0cm,
rightmargin=0cm,
linewidth=4pt,
innerbottommargin=5pt}

\mdfdefinestyle{rBox}{skipabove=\myskip,
skipbelow=\myskip,
roundcorner=5pt,
outerlinewidth=0,
nobreak=true,
backgroundcolor=maincolor!10,
linecolor=maincolor,
innerleftmargin=7pt,
innerrightmargin=7pt,
innertopmargin=7pt,
innerbottommargin=7pt,
leftmargin=0pt,
rightmargin=0pt,
linewidth=1pt
}

\mdfdefinestyle{sBox}{skipabove=\myskip,
skipbelow=\myskip,
nobreak=true,
hidealllines,
innerleftmargin=0pt,
innerrightmargin=0pt,
innertopmargin=0pt,
innerbottommargin=0pt,
leftmargin=0pt,
rightmargin=0pt,
linewidth=0pt
}


% surround theorems with boxes
\surroundwithmdframed[style=tBox]{theorem}
\surroundwithmdframed[style=dBox]{definition}
\surroundwithmdframed[style=eBox]{exercise}
\surroundwithmdframed[style=sBox]{example}

% use this to surround solutions with a samepage environment
%\newenvironment{solution}{\begin{samepage}\begin{solutionX}}{\end{solutionX}\end{samepage}}


%%%%%%%%%%%%%%%%%%%%%%%%%%%%%%%%%%%%%%%%%%%%%%%%%%%%%%%%%%%%%%%%%%%%%%%%%%%
% Widefigure and Objectives Environments
%%%%%%%%%%%%%%%%%%%%%%%%%%%%%%%%%%%%%%%%%%%%%%%%%%%%%%%%%%%%%%%%%%%%%%%%%%%
\usepackage{environ}


\newenvironment{widefigure}[1][htbp]{
\begin{figure}[#1]
  \begin{adjustwidth}{-2.00in}{-0in}
}
{
\end{adjustwidth}
\end{figure}
}


% Use this environment if you don't want objectives in the margins
% \NewEnviron{objectives}[1]{%
% \begin{adjustwidth}{-2.125in}{0in}

% \noindent{\sffamily\textbf{\Large Objectives}}

% \noindent%
% #1
% \vspace{-3mm}
% \begin{multicols}{2}
% \begin{itemize}[nosep,leftmargin=10.0pt]
%     \BODY
% \end{itemize}
% \end{multicols}
% \end{adjustwidth}

% \marginpar{\rule{1cm}{0in}}
% }


% Use this environment if you want objectives in the margins
\NewEnviron{objectives}[1]{%
  \marginpar{%
    \vspace{-0.1em}
  \noindent\textbf{\sffamily\Large Objectives}\\[2pt]
    #1
\begin{itemize}[nosep,leftmargin=10.0pt]
    \BODY
\end{itemize}
  \vspace{1em minus 1em}
  }
}



\let\oldsection\section
\renewcommand{\section}[1]{\oldsection{#1}\marginpar{\rule{1cm}{0in}}}


%%%%%%%%%%%%%%%%%%%%%%%%%%%%%%%%%%%%%%%%%%%%%%%%%%%%%%%%%%%%%%%%%%%%%%%%%%%
% Miscellaneous Packages
%%%%%%%%%%%%%%%%%%%%%%%%%%%%%%%%%%%%%%%%%%%%%%%%%%%%%%%%%%%%%%%%%%%%%%%%%%%


\usepackage{docmute} % for muting preamble of input files
% https://en.wikibooks.org/wiki/LaTeX/Page_Layout
% \usepackage[defaultlines=3,all]{nowidow}
\usepackage{graphicx} % Required for including pictures
\graphicspath{{pictures/}} % Specifies where pictures are stored
\usepackage{multicol}
\usepackage{float} % for H option in float placement
\usepackage{sidenotes}  % for marginfigure/margintable environment
\usepackage[font=small,labelfont={sf}]{caption}
\usepackage{subcaption}
\DeclareCaptionStyle{marginfigure}{font=footnotesize,labelfont={sf}}
\DeclareCaptionStyle{margintable}{font=footnotesize,labelfont={sf}}
\DeclareCaptionStyle{sidecaption}{labelfont={sf}}
\let\footnote\marginnote  %% marginpars in environments like align don't work
\usepackage{colortbl}
\usepackage{cancel}
\usepackage[bookmarksnumbered,linkbordercolor={1 1 1}]{hyperref} % for hyperlinks within document




%\input{glyphtounicode}
%\pdfgentounicode=1

\makeatother



